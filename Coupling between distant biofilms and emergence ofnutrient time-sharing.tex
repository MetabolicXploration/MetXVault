\textbf{Coupling between distant biofilms and emergence of nutrient time-sharing}

Jintao Liu,\(^{1}\) Rosa Martinez-Corral,\(^{2*}\) Arthur Prindle,\(^{1*}\) Dong-yeon D. Lee,\(^{1}\) Joseph Larkin,\(^{1}\) Marçal Gabalda-Sagarra,\(^{2}\) Jordi Garcia-Ojalvo,\(^{2}\) Gürol M. Süel\(^{1,3,4†}\)

\(^{1}\)Division of Biological Sciences, University of California, San Diego, CA 92093, USA. \(^{2}\)Department of Experimental and Health Sciences, Universitat Pompeu Fabra, 08003 Barcelona, Spain. \(^{3}\)San Diego Center for Systems Biology, University of California, San Diego, CA 92093, USA. \(^{4}\)Center for Microbiome Innovation, University of California, San Diego, CA 92093, USA.

*These authors contributed equally to this work.

†Corresponding author. E-mail: gsuel@ucsd.edu

Bacteria within communities can interact to organize their behavior. It remains unclear whether such interactions extend beyond a single community to coordinate the behavior of distant populations. We discovered that two \textit{Bacillus subtilis} biofilm communities undergoing metabolic oscillations become coupled through electrical signaling and synchronize their growth dynamics. Coupling increases competition by also synchronizing demand for limited nutrients. As predicted by mathematical modeling, we confirm that biofilms resolve this conflict by switching from in-phase to anti-phase oscillations. This results in time-sharing behavior where each community takes turns consuming nutrients. Time-sharing enables biofilms to counterintuitively increase growth under reduced nutrient supply. Distant biofilms can thus coordinate their behavior to resolve nutrient competition through time-sharing, a strategy used in engineered systems to allocate limited resources.

Biological systems often experience resource limitation (1–7). One strategy typically employed in engineered systems to cope with such challenges is known as time-sharing (8, 9), in which users take turns consuming resources. Time-sharing requires competing systems to vary their state in time and coordinate their dynamics. In general, coordination only arises from interactions such as communication among functional units, such as cells, to direct systems behavior. For example, bacteria within a population can communicate through various mechanisms, including quorum sensing and electrical cell-to-cell signaling mediated by ion channels (10–15). However, it remains unclear whether distinct populations of bacteria can act as functional units, and whether cell-to-cell interactions can extend to couple distant populations. Here we study whether two bacterial biofilm communities can coordinate their growth dynamics to engage in time-sharing and resolve competition for limited resources.

We studied \textit{Bacillus subtilis} biofilm communities that engage in collective growth-rate oscillations in response to glutamate starvation (5). Oscillations are driven by a spatially extended negative feedback loop, where growth of the biofilm results in glutamate stress within the interior, and this stress in turn interferes with biofilm growth (Fig. 1A and Fig. S1) (5). Coordination of these growth-rate oscillations within a biofilm are facilitated by potassium ion channel-mediated electrical cell-to-cell signaling (15). Since these long-range electrical signals have been shown to extend beyond the biofilm (16), we hypothesized that neighboring biofilms could potentially affect each other’s growth rate. Such biofilms would then also compete directly for nutrients in the shared environment. Therefore, biofilm pairs can be coupled through two basic mechanisms, namely communication and competition for nutrients (Fig. 1A).

To investigate coupling between biofilms, we used a large (3 mm × 3 mm × 6 μm) microfluidic chamber that can accommodate the growth of two oscillating biofilms separated by approximately 2 mm (Fig. 1B). Biofilms were cultured using standard MSgg biofilm-promoting media (17, 18) (glycerol and glutamate as main carbon and nitrogen sources, respectively) and at a steady flow rate (24 μm/s). We used time-lapse phase-contrast microscopy to directly measure colony expansion rate. Electrical signaling dynamics of each biofilm were measured using the fluorescent cationic dye Thioflavin T (ThT), which acts as a Nernstian voltage indicator of bacterial membrane potential (19) (Fig. 1C). As reported previously, growth rate and ThT oscillations are anti-correlated (Fig. 1, D and E) and can be used interchangeably to characterize biofilm dynamics. Our measurements revealed that two distant biofilms can exhibit synchronized oscillations in both growth rate and electrical signaling (Fig. 1, C to E, and movie S1). The average phase difference between oscillating biofilm pairs was \(0.06 \pm 0.07 \, \pi\) (s.d., \(n = 10\) experiments) and it persisted during the course of the experiment (~10 hours). The observed synchronization suggests not only that two distant biofilms interact, but also that collective oscillations in each of the two biofilms can become coupled (fig. S2, A and C).

We turned to mathematical modeling to examine how the interplay between competition and communication determines synchronization between two oscillating biofilms. Specifically, we modeled the biofilms as two coupled phase oscillators (20–24) to represent their growth dynamics (fig. S3 and supplementary text). We explicitly modeled competition for glutamate, which is used as a nitrogen source in the medium and is required for biofilm growth. The biofilms were also assumed to communicate through known electrical signaling during periods of metabolic stress, which can also couple their growth dynamics (5, 15). Furthermore, we assumed that communication increases with the concentration of glutamate in the medium, as the activity of the potassium ion channel underlying electrical signaling is regulated by glutamate availability (15). With these assumptions, the mathematical model could predict synchronization between two biofilms as a function of glutamate concentration and communication strength.

The model additionally predicted anti-phase oscillations at lower glutamate concentrations in the medium due to enhanced competition (Fig. 2A and fig. S4C). In particular, enhanced competition increases the tendency of the biofilms to halt their growth. Stochastically, one of the biofilms will start this process before the other. This will allow the second biofilm to postpone halting its own growth, thus increasing the phase difference between the biofilms. This process destabilizes the in-phase dynamics, leading to anti-phase oscillations (fig. S4C). In contrast, higher concentrations of glutamate promote in-phase oscillations through stronger communication (fig. S3). Here enhanced communication forces the two biofilms to share their stress state, leading to synchronized phases (fig. S4B). These results show that, depending on the balance between competition and communication, the system of coupled biofilms is predicted to have two attractor states corresponding to in-phase or anti-phase oscillations.

We experimentally tested these predictions by measuring the synchronization between pairs of oscillating biofilms growing under a steady supply of regular (30 mM) or reduced (by 25\%) concentration of glutamate. As predicted, we found in-phase oscillations at regular concentrations of glutamate [Fig. 2B (top), figs. S5B and S6, and movie S2] and approximately anti-phase oscillations at lower glutamate concentrations [Fig. 2B (bottom); figs. S2, B to C, S5A, and S6; and movie S3]. Consequently, glutamate limitation is sufficient to induce changes in the synchronization between two biofilms.

The model also predicted that the transition from in-phase to anti-phase oscillations depends on communication strength [Fig. 2A (right), fig. S4, and supplementary text]. Electrical signaling in \textit{B. subtilis} biofilms is mediated by the YugO ion channel that is gated by a TrkA domain (25–27). To test the effect of communication strength, we used a previously characterized truncated YugO potassium ion channel that lacks the TrkA gating domain. While a complete deletion of the YugO ion channel interferes with biofilm formation (15, 25), the truncated version, lacking the TrkA gating domain, simply decreases the transmission efficiency of electrical signaling without abolishing it completely (15). Such reduced communication within a biofilm may also reduce synchronization between two biofilms. The model predicted that for lower communication strengths, higher glutamate concentrations are needed to reach in-phase oscillations between biofilms [Fig. 2A (right) and fig. S4E]. Indeed, \Delta\textit{trkA} mutant biofilms did not synchronize at regular or 50\% increased concentration of glutamate [Fig. 2C (bottom) and figs. S5C, S6, and S7]. As predicted, when glutamate concentrations were doubled we observed in-phase oscillations between mutant biofilms [Fig. 2C (top) and figs. S5D and S6]. Thus the transition from in-phase to anti-phase oscillations also depends on the communication between biofilms.

The model predicted that the transition to anti-phase dynamics depends on competition strength as well [Fig. 2A (left)]. Accordingly, we constructed a mutant strain that cannot synthesize its own glutamate and thus has a higher demand for externally supplied glutamate. Specifically, we disrupted the \textit{gltA} gene encoding glutamate synthase (\Delta\textit{gltA}). This enzyme allows cells to synthesize two glutamate molecules from combining one molecule of glutamine and one of alpha-ketoglutarate (fig. S8) (28). Consequently, two \Delta\textit{gltA} biofilms should experience higher competition for glutamate compared to wild-type biofilms. As predicted, \Delta\textit{gltA} biofilms failed to synchronize under baseline glutamate concentrations [Fig. 2D (bottom)], for which the wild-type biofilms readily synchronized. Glutamate concentration had to be increased by 25\% to generate the predicted in-phase oscillations between \Delta\textit{gltA} biofilms [Fig. 2D (top)]. The synchronization dynamics between two biofilms thus also depend on nutrient competition. Together, perturbations to communication and competition strength between biofilms confirm the mathematically predicted 3-dimensional phase diagram that defines the regions where two biofilms oscillate either in-phase or anti-phase (Fig. 2E).

We tested whether the observed anti-phase dynamics that would allow time-sharing, provided a benefit for biofilm growth. In the case of in-phase oscillations, each biofilm would effectively obtain only half of the available resources (resource-splitting) during its growth phase (Fig. 3A). In contrast, time-sharing would allow each biofilm to take turns in having access to all the resources supplied at a constant rate (Fig. 3A). Counterintuitively, biofilm growth would thus increase when nutrient supply is low, because reduction in concentration of glutamate promotes the transition to time-sharing. Our mathematical model indeed predicted greater biofilm growth when glutamate concentrations were reduced by 25\% (Fig. 3B and supplementary text).

We experimentally tested the prediction that reduced concentrations of glutamate would improve biofilm growth through time-sharing. We first confirmed the basic expectation that two in-phase biofilms compete for nutrients by showing that their time-averaged growth rates are slower compared to that of a single biofilm [Fig. 3C (1× glutamate) and fig. S9]. We also verified that the growth rate of a single biofilm is slower at reduced concentrations of glutamate [Fig. 3C (gray line)]. In contrast to a single biofilm, two biofilms growing in lower glutamate concentrations had a faster average growth rate than two biofilms growing at higher glutamate concentrations [Fig. 3C (black line)]. Observing a faster growth rate for biofilms growing under lower glutamate concentrations is surprising, yet can be explained by the switch from resource-splitting to time-sharing. At higher glutamate concentrations, in-phase nutrient consumption leads to direct competition and resource-splitting. In contrast, at lower concentrations of glutamate biofilms consume nutrients at different times, and the resulting time-sharing of resources promotes growth. Accordingly, the observed difference in growth rates is accounted for by the phase difference between the biofilms (Fig. 3D). These results demonstrate the benefit of time-sharing and reveal that the average growth rate of biofilms is not exclusively defined by absolute nutrient concentrations but also by the resource sharing strategy between biofilms.

Our data show that bacteria appear to resolve conflicts that arise from competition for resources between distant communities. It remains unclear why biofilms would synchronize their growth in the first place. The fitness benefit of communication among bacteria that allows coordination within a biofilm may bring with it the cost of synchronization between communities. This communication cost may not be restrictive when nutrient concentrations are sufficiently high, but could be detrimental when the concentration of nutrients is too low. Our results show that glutamate-dependent modulation of competition and communication allows biofilms to alleviate this cost by engaging in time-sharing when nutrients become more limited. It remains to be determined in future studies whether two biofilms formed by different bacterial species can also engage in time-sharing, or if this behavior is limited to kin populations, due to evolutionary incentives. From another perspective, time-sharing is a common strategy employed in computer science to share computing resources between users (8, 9). This connection between engineered and natural systems may in the future allow us to utilize time-sharing in synthetic biology applications focusing not only on interactions within, but also between communities.

Fig. 1. Distant biofilms synchronize their growth dynamics. (A) Individual biofilms undergo metabolic oscillations that periodically halt growth. The metabolic oscillation is facilitated by electrical communication, which can extend beyond one biofilm and couple distant biofilms (cyan signals). In addition, two biofilms can also be coupled through competition for nutrients (red arrows). (B) Schematic depicting two biofilms grown on the two sides of a microfluidic chamber, with steady media flow. Purple and orange rectangles represent regions shown in (C). (C) Filmstrip showing the edges of a biofilm pair over time. Cyan indicates fluorescence of Thioflavin T (ThT), a cationic fluorescent dye that reports membrane potential within the biofilm. Scale bar, 50 μm. (D) Growth rate oscillation measured by expansion speed of biofilm edges shown in (C). (E) Membrane potential oscillation measured from the mean ThT fluorescence at biofilm edges shown in (C).

Fig. 2. Synchronization between biofilms is governed by communication and competition. (A) Phase diagrams computed using a mathematical model of coupled phase oscillators show in-phase (green-shaded regions) and anti-phase (red-shaded regions) oscillations. The colored dots indicate the experimental validations shown in the following panels. (B to D) Experimental results of wild-type (B), \(\Delta trkA\) (C), and \(\Delta gltA\) (D) biofilms. For each strain, the biofilm pairs showed in-phase (approximately 0 phase difference) oscillations at high glutamate concentrations, and anti-phase (approximately \(\pi\) phase difference) oscillations at low glutamate concentrations. In each panel, the filmstrip shows the membrane potential oscillation of a representative biofilm pair, with corresponding time traces (color coded by biofilm). The scatterplots show membrane potential of biofilm pairs against each other for individual time points (\(n=3\) experiments per plot). (E) Three-dimensional phase diagram summarizing model prediction and experimental validations. The gray-shaded surface depicts the boundary between regions of in-phase and anti-phase oscillations. The black and cyan lines indicate the corresponding two-dimensional phase diagram boundaries shown in (A).

Fig. 3. Time-sharing resolves nutrient competition between biofilms. (A) Anti-phase oscillations (time-sharing) allow each biofilm to take turns accessing the full quantity of supplied nutrients during its growth phase. In contrast, in-phase oscillations (resource-splitting) only allow half of the supplied nutrients to each biofilm during its growth phase. (B) Model prediction and (C) experimental validation of average growth rate for single biofilm (grey line) and biofilm pair (black line) at different glutamate concentrations. (D) Biofilm growth rate is determined by the phase difference between biofilm pairs. Pairs of wild-type (solid line), \(\Delta trkA\) (dashed line), and \(\Delta gltA\) (dotted line) biofilms all showed faster growth with anti-phase oscillations (time-sharing) compared to in-phase oscillations (resource-splitting). The blue shading indicates glutamate concentration. Error bars represent s.e.m., \(n = 3\) experiments.